\documentclass{article}
%!TeX spellcheck = en_US 
\usepackage[left=2.5cm, right=2.5cm, top=2cm, bottom=2.5cm]{geometry}
\usepackage{xcolor}
\usepackage{mdframed}
\usepackage{gensymb}
\usepackage[colorlinks, linkcolor = black, citecolor = black, filecolor = black, urlcolor = blue]{hyperref}
\setlength{\parindent}{0pt} 
\usepackage{enumitem}
\usepackage{graphicx}
\usepackage{float}
\usepackage{mathrsfs}
\usepackage{amssymb}
\usepackage{amsmath}

\begin{document}
\title{GEO2310 Meteorology | Exercise 13.03.24\\ Planetary Boundary Layer, Part 1 }
\date{}
\maketitle

\vspace{-1.3cm}

\section{Turbulence and Static Stability}
\begin{enumerate}[label=(\alph*)]
    \item What are characteristics of turbulence? %\textit{Chaotic, irregular, rotational, dissipative}
    \item Which mechanisms create or destroy turbulence? %\textit{Turbulence can be created mechanically (due to shear) and thermally (due to buoyancy). Negative buoyancy destroys turbulence.}
    \item Explain the terms "static stability" and "lapse rate". Draw three vertical profiles representing stable, neutral and unstable conditions. What do we know about the lapse rates under these conditions? %\textit{To determine static stability we compare the temperature of an air parcel with the background temperature. If the parcel is warmer (= less dense) it rises (unstable), if it has the same temperature it does not move (neutral) and if it colder (= denser) it sinks (stable). Lapse rate describes the vertical temperature change in K/m or $^\circ$C/m. The dry adiabatic lapse rate is $\Gamma_d = 9.8^\circ$C/km. Stable: $\Gamma<\Gamma_d$, neutral: $\Gamma=\Gamma_d$, unstable: $\Gamma>\Gamma_d$.}
    \item What is potential temperature? Write down the equation and explain how it is connected to static stability. %\textit{    The temperature a parcel would have, if it is brought adiabatically (i.e., without heat exchange) to a reference pressure of $p_0$: $\theta = T \left(   \dfrac{p_0}{p}   \right)^{R/c_p}$ \\
	%The vertical potential temperature gradient is related to static stability:
    %\begin{align}
%        \frac{\partial \theta}{\partial z} \begin{cases}
%            > 0 \rightarrow stable\\
%            = 0 \rightarrow neutral\\
%            < 0 \rightarrow unstable
%        \end{cases}
%    \end{align}
%    }
\end{enumerate}

\section{Statistical Description of Turbulence}
\begin{enumerate}[label=(\alph*)]
    \item What is Reynolds decomposition? Why do we describe turbulence statistically? %\textit{Reynolds decomposition: $x = \overline{x} + x'$. We describe turbulence statistically because of its random behaviour.}
    \item Describe the concept of Kolmogorovs energy cascade and draw (schematically) the corresponding energy spectrum. %\textit{Large eddies dissipate energy by decaying into smaller and smaller eddies until molecular viscosity. }
    \item Turbulence intensity can be described by using the variance, e.g. $\sigma_u^2,\sigma_v^2,\sigma_w^2$ corresponding to the variance of streamwise, crosswise and vertical wind speed. Use them, to define the terms "stationary", "homogeneous" and "isotropic" turbulence. How can you interprete these three flow characteristics? 
    %\textit{homogeneous: $\sigma_x^2$ is constant in space (horizontally) \\
    %stationary:  $\sigma_x^2$ is constant in time \\
    %isotropic: $\sigma_u^2 = \sigma_v^2 = \sigma_w^2$, turbulence intensity is the same in all three dimensions.}
    \item What is turbulent kinetic energy (TKE)? Write down the TKE equation and explain the terms. Which terms are sources or sinks of TKE? %\textit{$TKE := 0.5 (\sigma_u^2 + \sigma_v^2 + \sigma_w^2)$ \\
    %TKE equation: \\
    %$\frac{\partial TKE}{\partial t} = $ Advection + Mechanical Production + Buoyancy Generation + Transport + Dissipation \\
    %Therein: dissipation is always negative, mechanical production positive, all other terms can be either positive (= create TKE) or negative (= destroy TKE).}
\end{enumerate}

\section{Turbulence Closures and Logarithmic Wind Profile}
\begin{enumerate}[label=(\alph*)]
    \item What is a parametrization? Why (and when) do we need to parametrize turbulence? %\textit{A parametrization is a way to describe the net effect of non-resolved ("sub-grid scale") processes on the large scale resolved motions. Turbulence is a sub-grid scale process.}
    \item An example for a first-order closure is given by:
    \begin{align}
        \overline{w'\theta'} = - K_H \, \frac{\partial \overline{\theta}}{\partial z}
    \end{align}
    Explain both sides of the equation and how they are connected (e.g., with a sketch). What does $K_H$ represent? Why is it called first-order closure? %    \textit{The left hand side represents the vertical heat flux (correlation of vertical wind speed and potential temperature) and the right hand side the mean vertical gradient of potential temperature (= measure of stability). $K_H$ is a diffusivity coefficient. The equation states, that the vertical heat flux tries to balance the temperature gradient. It is called first-order closure because the highest used moment on the right hand side (= in the parametrization/closure) is of first-order (i.e., mean value).}
    \item Sketch and explain the logarithmic wind profile. Write down the equation and name all variables. %\textit{    The logarithmic wind profile describes the increase in wind speed with height (derived for neutral stratification):
%    $$\overline{v}(z) = \frac{u_*}{\kappa} \ln \left( \frac{z}{z_0}\right)$$
%    $u_*$ is the friction velocity, $\kappa = 0.4$ a constant and $z_0$ the roughness length.}
%    \item What is surface roughness length? What are typical roughness lengths for e.g. lakes, small vegetation, forests, cities? \textit{Surface roughness length is a concept to describe the roughness of a surface with a length scale (i.e., in m) and is always smaller than the physical height. Typical values reach from 10$^{-4}$ for lakes to 0.2 for tall forests or cities.}
    
\end{enumerate}


\end{document}